\chapter{Introduction}

Les modèles déterministes et les modèles probabilistes. Les modèles déterministes sont plus faciles à étudier, mais ils ne reflètent pas toujours la réalité. Les modèles probabilistes, quant à eux, sont une façon naturelle de modéliser l’évolution d’une épidémie, car chaque individu a une certaine probabilité d’être infecté par la maladie.

L’étude de ces problèmes stochastiques est importante pour déterminer si, quand la taille de la population augmente, ils convergent vers un problème déterministe. Nous nous intéresserons plus particulièrement aux épidémies qui se transmettent d’individu en individu et confèrent une immunité aux contacts suivants en cas de rétablissement. Les maladies infantiles, sexuellement transmissibles ou même la grippe correspondent à ce genre de pathologie. On appelle ce type de modèle une épidémie SIR (Susceptible, Infected, Recovered). La population est ainsi donc séparée en trois catégories. Nous avons les susceptibles qui sont les individus sains et sensibles à la maladie, les infectés qui sont atteints par l’épidémie et pouvant donc la transmettre pendant une certaine période de contagion, et ceux qui sont immunisés contre une nouvelle infection après avoir été infectés.