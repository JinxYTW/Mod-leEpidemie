\begin{frame}
        \frametitle{Modéle SIR}

        \begin{block}{Définition}

                Modéle divisé en trois groupe :
                \begin{itemize}
                        \item Susceptible : personnes saines qui n'ont pas encore contracté la maladie
                        \item Infectés : les malades qui transemettent activement la maladie
                        \item Rétablis : les guéris ou mort, qui ne peuvent plus contracté la maladie.
                \end{itemize}

        \end{block}
\end{frame}

\begin{frame}
        \frametitle{Modéle SIR}

        \begin{alertblock}{Equation}

                $$ \frac{dS}{dt} = -\frac{\beta SI}{N} \qquad \frac{dI}{dt} = \frac{\beta SI}{N} - \gamma I \qquad \frac{dR}{dt} = \gamma I $$

                \begin{itemize}
                        \item $N$ : taille totale de la population
                        \item $\beta$ : taux de contact
                        \item $\gamma$ : taux de guérison.
                \end{itemize}

        \end{alertblock}
\end{frame}

\begin{frame}
        \frametitle{Modéle SIR}

        \centering
        \includegraphics[scale=0.25]{sir_deterministe}

        \begin{itemize}
                \item $\gamma = 0.1$
                \item $\beta = 0.48$
                \item $N = 10000$
                \item $I_0 = 1$
        \end{itemize}

\end{frame}

\begin{frame}
        \frametitle{Nombre de reproduction de base ($R_0$)}

        Nombre attendu d’infections générées par un individu infectieux dans une grande population susceptible.

        \begin{alertblock}{SIR}
                $$ R_0 =  \lambda i $$
        \end{alertblock}

        \begin{itemize}
                \item $i$ : durée moyenne de la période infectieuse
                \item $\lambda$ : intensité du processus de Poisson homogène qui modélise les contacts entre les individus infectés et les individus susceptibles
        \end{itemize}
\end{frame}
