\subsection{Le modèle SIR}

\begin{frame}
        \frametitle{Modéle SIR}

        \begin{block}{Définition}

                Modéle divisé en trois groupe :
                \begin{itemize}
                        \item Susceptible : personnes saines qui n'ont pas encore contracté la maladie
                        \item Infectés : les malades qui transemettent activement la maladie
                        \item Rétablis : les guéris ou mort, qui ne peuvent plus contracté la maladie.
                \end{itemize}

        \end{block}
\end{frame}

\begin{frame}
        \frametitle{Modéle SIR}

        \begin{alertblock}{Propriéter}

                $$\frac{dS}{dt} = -\frac{\beta SI}{N} \qquad \frac{dI}{dt} = \frac{\beta SI}{N} - \gamma I \qquad \frac{dR}{dt} = \gamma I$$

                \begin{itemize}
                        \item $N$ : taille totale de la population
                        \item $\beta$ : taux de contact
                        \item $\gamma$ : taux de guérison.
                \end{itemize}

        \end{alertblock}
\end{frame}
