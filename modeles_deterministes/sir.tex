\section{Le modèle SIR}
        \subsection{Histoire et amélioration}

	Le modèle SIR est un modèle mathématique simple qui permet de modéliser la propagation d'une maladie infectieuse dans une population. Le modèle divise la population en trois groupes : les individus susceptibles d'être infectés (S), les individus infectés (I) et les individus guéris ou retirés (R). Les individus passent d'un groupe à l'autre en fonction de leur état de santé. Le modèle est décrit par les équations différentielles suivantes :
$$\frac{dS}{dt} = -\frac{\beta SI}{N}$$

$$\frac{dI}{dt} = \frac{\beta SI}{N} - \gamma I$$

$$\frac{dR}{dt} = \gamma I$$
où S, I et R représentent respectivement le nombre de personnes dans chaque groupe, N est la taille totale de la population, $\beta$ est le taux de contact et $\gamma$ est le taux de guérison. Le modèle de Reed et Frost est un autre modèle mathématique simple qui permet de modéliser la propagation d'une maladie infectieuse dans une population. Dans ce modèle, les individus sont divisés en trois groupes : les individus susceptibles d'être infectés (S), les individus infectés (I) et les individus retirés (R). Le modèle suppose que chaque individu infecté a une probabilité fixe de transmettre la maladie à un individu susceptible chaque semaine. Le nombre de reproduction de base R0 est défini comme le nombre moyen de susceptibles qu'un individu infecté infecte au début de l'épidémie, lorsque presque toute la population est susceptible.

Le nombre de reproduction de base, R0, est défini comme le nombre attendu d'infections générées par un individu infectieux dans une grande population susceptible. Pour le modèle épidémique SIR standard que nous avons décrit précédemment, R0 est défini comme $\lambda i$ où i est la durée moyenne de la période infectieuse et $\lambda$ est l'intensité du processus de Poisson homogène qui modélise les contacts entre les individus infectés et les individus susceptibles.
