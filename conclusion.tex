\chapter{Conclusion}

\begin{theorem}[Loi faible des grands nombres]
 Soit $(X_n)_{n \in \boldsymbol{N}}$ une suite de variables aléatoires indépendantes et identiquement distribuées de même loi que X (loi mère). Si $\mathbb{E}[|X|] < +\infty$, alors $\overline{X_n} \rightarrow \mathbb{E}[X]$
\end{theorem}


On ne rappelle pas la loi faible des grands nombres pour rien.\\
En effet, elle peut être observée dans les modèles SIR déterministes et stochastiques.\\
Plus précisément, on peut montrer que le modèle stochastique converge vers le modèle déterministe lorsque la taille de la population tend vers l'infini. Ainsi, on peut en conclure que la faible loi des grands nombres est un concept qui fournit une justification mathématique à l'utilisation du modèle déterministe pour approcher le comportement des modèles stochastiques parmi une grande population, telle qu'une épidémie mondiale.

\begin{figure}[h]
\centering
\includegraphics[width=0.5\textwidth]{figs/sir_loi_grands_nombres.png}
\caption{Loi faible des grands nombres appliqués au modèle stochastique SIR}
\label{fig:loi_grands_nombres}
\end{figure}