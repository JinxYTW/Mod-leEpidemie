\subsection{Bartlett}

\begin{frame}
    \frametitle{Quelques rappels}
    \begin{block}{Processus de Poisson}
        \begin{itemize}
            \item \textbf{Incrément constant}
            \item \textbf{Evénement unique}
            \item \textbf{Indépendance}
            \item \textbf{Invariance}
            \item \textbf{Loi de Poisson}
        \end{itemize}
    \end{block}
\end{frame}

\begin{frame}
    \frametitle{Quelques rappels}
    \begin{block}{Loi de Poisson}
        \begin{itemize}
            \item $ P(X=k) = \exp(-\lambda)\frac{\lambda^k}{k!} $
        \end{itemize}
    \end{block}
\end{frame}

\begin{frame}
    \frametitle{Bartlett}

    \begin{block}{Histoire}
        En 1949, le statisticien Maurice S.Bartlett pose ce qui sera le modèle épidémiologique standard.
    \end{block}

    \begin{itemize}
        \item La population se compose de m infectés et n susceptibles
        \item La période de contagion suit une loi I (moyenne $i$, variance $\sigma^2$)
        \item La probabilité de rencontrer un individu suit un processus de Poisson d’intensité $\frac{\lambda}{n}$
        \item Cette épidémie est notée $E_{n,m}(\lambda, I)$
    \end{itemize}
\end{frame}

