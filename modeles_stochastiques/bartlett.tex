\section{Bartlett}

C’est en 1949 que Bartlett propose ce qui sera le modèle épidémiologique standard. Au temps
initial, la population se compose de m infectés et n susceptibles. Chaque infecté a une période de
contagion qui suit une loi I (moyenne $i$ variance $\sigma^2$ ) indépendante de celle des autres. Et pendant
cette période il rencontre un autre individu donné suivant les instants de saut d’un processus de
Poisson d’intensité $\frac{\lambda}{n}$. Tous ces processus sont indépendants entre eux, ainsi que leur périodes de
contagion. Ainsi, pendant un temps $dt$, la probabilité qu’un infecté rencontre un individu initialement
susceptible vaut $\lambda$. On appellera $E_{n,m}(\lambda, I)$ une telle épidémie. Un des paramètres importants dans
ce modèle est le nombre d’infections attendues par un individu (Basic reproduction number) $R_0 =
\lambda i$. On verra qu’en fonction de sa valeur, l’épidémie se propage à toute la population ou reste
confinée à un plus petit nombre d’individus.

\subsection{Loi de poisson}

Une variable aléatoire X suit une loi de poisson si :
$$ P(X=k) = \exp(-\lambda \frac{\lambda^k}{k!}) $$
Cette loi est dite des événements rares : elle est souvent utilisé pour modéliser le nombre d'occurences d'un phénoménes dans un temps donnés, le phénoméne devant être rare dans un temps court.
