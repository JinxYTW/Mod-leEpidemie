\section{Les améliorations ammenées par le modèle de Bartlett}

\subsection{Quelques rappels}
Afin d'introduire et permettre une meilleure compréhension de ce qui va suivre, permettons-nous quelques rappels.

\subsubsection{Processus de Poisson}
Un processus de Poisson est un type de processus stochastique qui modélise le nombre d'événements se produisant dans un intervalle de temps donné ou sur un intervalle spatial donné.
De façon plus consciencieuse, elle se définit comme suit :

Soit $(S_1, ..., S_n)$ une suite de variables aléatoires exponentielles indépendantes de paramètre $\lambda$. Posons $T_n = \sum^n_1 S_i (T_0 = 0)$ et $\forall t \in \boldsymbol{R}$ :

\begin{center}
    $$ N_t = \sum_{n \geq 0} \boldsymbol{1}_{\{T_n \leq t \}} $$
\end{center}

La famille aléatoire $(N_t)$ s'appelle le \textbf{processus de Poisson standard issu de 0 d'intensité $\lambda$}. Ce processus compte alors le nombre de variables $T_i$ dans $[0, t]$.

Un processus de Poisson à alors les propriéter suivantes :
	\begin{itemize}[label=$\bullet$]
    	\item \textbf{Incrément constant} : $\forall \omega$, $t \mapsto N_t$ est croissante, continue à droite, constante par morceaux et ne croît que par sauts de 1, avec  $N_t$ finie presque sûrement.
    	\item \textbf{Evénement unique} : si $N_{t_-}(\omega)$ désigne la limite à gauche de $N_t(\omega)$ aux point t alors $\forall t$, $N_t = N_{t_-}$ presque sûrement.  	
    	\item \textbf{Loi de Poisson} : $\forall t$, $N_t$ suit une loi de Poisson de paramètre $\lambda t$.
    	\item \textbf{Indépendance} : $\forall (t_1, t_2, ..., t_n)$ tels que $0 < t_1 < t_2 < ... < t_n$ les variables $N_{t_1}$,$N_{t_2} - N_{t_1}$, ..., $N_{t_n} - N_{t_{n-1}}$ sont indépendantes.
    	 \item \textbf{Invariance} : si $s < t$, la loi de $N_t - N_s$ est la même que celle de $N_{t - s}$.
    	 \end{itemize}

\subsubsection{Loi de poisson}
On rappelle aussi que si une variable aléatoire X suit une loi de poisson alors on a :\\
$$ P(X=k) = \exp(-\lambda)\frac{\lambda^k}{k!} $$

    	 
\subsection{Le modèle de Bartlett}
Les rappels étant maintenant terminés, revenons-en au modèle de Bartlett.\\
C’est en 1949 que le statisticien le statisticien Maurice S. Bartlett propose ce qui sera le modèle épidémiologique standard.Celui-ci sera défini ainsi:\\
Au temps initial, la population se compose de m infectés et n susceptibles. Chaque infecté a une période de contagion qui suit une loi I (moyenne $i$ variance $\sigma^2$ ) indépendante de celle des autres.\\
Et pendant cette période il rencontre un autre individu donné suivant les instants de saut d’un processus de Poisson d’intensité $\frac{\lambda}{n}$. Ainsi, pendant un temps $dt$, la probabilité qu’un infecté rencontre un individu initialement susceptible vaut $\lambda$. \\
On appellera $E_{n,m}(\lambda, I)$ une telle épidémie.


