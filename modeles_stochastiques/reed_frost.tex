\section{Reed et Forst}

Cependant, le modèle SIR dans l’état que nous avons donné, possède certaines incohérence vis à vis
de la réalité. En effet, il suppose que chaque individu de la population a une chance égale d'entrer
en contact avec tous les autres individus, et que la chance de propager l’épidémie est, elle aussi uniforme.
C’est ici qu’intervient le modèle stochastique de Reed et Frost présenté en 1928, qui redéfinit le
modèele cité précedemment. S'ils sont infectés, ils deviennent d'abord infectieux pendant un certain
temps, puis se rétablissent et deviennent immunisés. Le modèle est généralement utilisé avec une
dynamique à temps discret, où la période infectieuse est courte et précédée d'une période de latence
plus longue. Les nouvelles infections se produisent par génération, séparées par la période de
latence. Les probabilités dans chaque génération dépendent de l'état de l'épidémie dans la
génération précédente, et sont spécifiées par des probabilités binomiales.

\begin{align}
P(Y_{j+1} = y_{j+1} | X_0 = x_0, Y_0 = y_0, ..., X_j = x_j, Y_j = y_j) &= P(Y_{j+1} = y_{j+1} | X_j = x_j, Y_j = y_j) \\
&= \binom{x_j}{y_{j+1}}(1 - q^{y_j})^{y_{j+1}}(q^{y_j})^{x_j - y_{j+1}}
\end{align}

\subsection{Loi Binomiale}

La loi Binomiale est la loi suivis par la variable aléatoire S qui compte le nombre de succés de n expériences de Bernouli consécutives et indépendantes.
La loi de Bernouli est la loi suivis par l'expérience qui peut résussir avec une probabilités p

