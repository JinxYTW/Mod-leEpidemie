\subsection{La construction de Selke}

\begin{frame}
    \frametitle{Définitions}

    \begin{block}{Construction de Selke}
        On assigne à chaque individu sain une résistance à l'épidémie, et la totalité des infectés définissent alors une pression de l'épidémie. On compare alors cette pression à la résistance de chaque individu sain pour savoir si ils deviennent infectés ou reste sains.
    \end{block}

    \begin{itemize}
        \item On utilise des chaînes de Markov pour modéliser l'épidémie au fil du temps
        \item Le nombre d'infections dans un temps donné est obtenu grâce à un processus de poisson
    \end{itemize}
\end{frame}

\begin{frame}
        \frametitle{Nombre de reproduction de base ($R_0$)}

        Nombre attendu d’infection générée par un individu infectieux dans une grande population susceptible.

        \begin{alertblock}{SIR}
                $$ R_0 =  \lambda i $$
        \end{alertblock}

        \begin{itemize}
                \item $i$ : durée moyenne de la période infectieuse
                \item $\lambda$ : intensité du processus de Poisson homogène qui modélise les contacts entre les individus infectés et les individus susceptibles
        \end{itemize}
\end{frame}

\begin{frame}
    \frametitle{Propriété}

    \begin{block}{Taille finale}
        $$ 1 - \tau = e - R_0\tau $$
    \end{block}

    \begin{itemize}
        \item $\tau$ est la proportion de la population qui sera infectée à la fin de l’épidémie
    \end{itemize}
\end{frame}


\begin{frame}
    \frametitle{Chaine de Markov}

    \begin{block}{Définition}
        Une chaîne de Markov est un processus aléatoire à temps discret dont la principale caractéristique est l’absence de mémoire et l’existence de probabilités de transition entre les états de la chaîne. Cela signifie que seul l’état actuel du processus a une influence sur l’état qui va suivre.
    \end{block}


\end{frame}

\begin{frame}
    \frametitle{Propriété}

        En considérant S un ensemble fini (ou dénombrable) et $(X_n)_{n \in \mathbb{N}}$ une suite de variables aléatoires définies sur le même univers $\Omega$ munit de la mesure de probabilité $\mathbb{P}$ et à valeurs dans S. $(X_n)$ chaîne de Markov homogéne si :

        \begin{itemize}
            \item \textbf{Propriété de Markov} : $\forall n \in \mathbb{N}$ et $(x_i)^n \in S^n (i \in [[0, n+1]])$ \\
            $$ \mathbb{P}(X_{n+1} = x_{n+1} | X_{0:n} = x_{0:n}) = \mathbb{P}(X_{n+1} = x_{n+1} | X_n = x_n) $$
            \item \textbf{Homogénéité} : $\mathbb{P}(X_{n+1} = y | X_n = x)$ ne dépend pas de n, $\forall (x, y) \in S^2$. On note alors $p(x, y)$ cette probabilité.
        \end{itemize}
\end{frame}
