\subsection{La construction de Selke}

\begin{frame}
    \frametitle{Définition}
    \begin{block}{Construction de Selke}
        En 1983,Sellke propose l'idée d'affecter à chaque individu une capacité à résister à l'épidémie,ainsi que l'ajout d'une pression épidémique.
    \end{block}

    \begin{itemize}
        \item Des chaînes de Markov pour modéliser l'épidémie au fil du temps
        \item Un processus de Poisson pour modéliser le nombre d'infectés
    \end{itemize}
\end{frame}


\begin{frame}
        \frametitle{Nombre de reproduction de base ($R_0$)}

        Nombre attendu d’infection générée par un individu infectieux dans une grande population susceptible.

        \begin{alertblock}{SIR}
                $$ R_0 =  \lambda i $$
        \end{alertblock}

        \begin{itemize}
                \item $i$ : durée moyenne de la période infectieuse
                \item $\lambda$ : intensité du processus de Poisson homogène modélisant les contacts entre les individus infectés et les individus susceptibles
        \end{itemize}
\end{frame}

\begin{frame}
    \frametitle{Propriété}

    \begin{block}{Taille finale}
        $$ 1 - \tau = e - R_0\tau $$
    \end{block}

    \begin{itemize}
        \item $\tau$ est la proportion de la population qui sera infectée à la fin de l’épidémie
    \end{itemize}
\end{frame}


\begin{frame}
    \frametitle{Chaine de Markov}

    \begin{block}{Définition}
        Une chaîne de Markov est un processus aléatoire à temps discret dont la principale caractéristique est l’absence de mémoire et l’existence de probabilités de transition entre les états de la chaîne.
    \end{block}
\end{frame}

\begin{frame}
    \frametitle{Propriété}

        Soient S un ensemble fini (ou dénombrable) et $(X_n)_{n \in \mathbb{N}}$ une suite de variables aléatoires définies sur le même univers $\Omega$ munit de la mesure de probabilité $\mathbb{P}$ et à valeurs dans S. $(X_n)$ est une chaîne de Markov  si :

        \begin{itemize}
            \item \textbf{Propriété de Markov} : $\forall n \in \mathbb{N}$ et $(x_i)^n \in S^n (i \in [[0, n+1]])$ \\
            $$ \mathbb{P}(X_{n+1} = x_{n+1} | X_{0:n} = x_{0:n}) = \mathbb{P}(X_{n+1} = x_{n+1} | X_n = x_n) $$
            \item \textbf{Homogénéité} : $\mathbb{P}(X_{n+1} = y | X_n = x)$ ne dépend pas de n, $\forall (x, y) \in S^2$. On note alors $p(x, y)$ cette probabilité.
        \end{itemize}
\end{frame}

\begin{frame}
	\frametitle{Définition du modèle}
	\begin{itemize}
	\item Les infectés initiaux seront notés $-(m-1), -(m-2), ...  -1, 0$
	\item Les susceptibles de 1 à $n$
	\item  A chaque susceptible on associe aussi des seuils $Q_1, Q_n$
	\end{itemize}
\end{frame}

\begin{frame}
	\frametitle{Formule}
	
	\begin{block}{Taille finale de l'épidémie}
          $ Z = \min \{\frac{\lambda}{n} \sum^i_{j = - (m-1)} I_j \} $
	\end{block}

\end{frame}

\begin{frame}
	\frametitle{Formule}
	\begin{block}{Identité de Wald}
          $\mathbb{E}[ \frac{ e^{-\theta P(\infty)} }{ \Phi( \frac{\lambda \theta}{n})^{Z + m} } ] = 1 $
	\end{block}	
\end{frame}

\begin{frame}
	\frametitle{Démonstration}
	\begin{itemize}
	\item $ \Phi( \frac{\lambda \theta}{n})^{n + m} = \mathbb{E}[e^{-(\frac{\lambda \theta}{n}) \sum^n_{j = - (m-1)} I_j }]$
	\item $ \mathbb{E}[e^{-(\frac{\lambda \theta}{n}) \sum^i_{j = - (m-1)} I_j }]= \mathbb{E}[e^{- \theta(Z+\frac{\lambda}{n} \sum^n_{j = Z+1} I_j }]$
	\item $\mathbb{E}[e^{- \theta(Z+\frac{\lambda}{n} \sum^n_{j = Z+1} I_j }]=\mathbb{E}[e^{- \theta Z}\Phi( \frac{\lambda \theta}{n})^{n - Z}]$
	\item $\mathbb{E}[ \frac{ e^{-\theta P(\infty)} }{ \Phi( \frac{\lambda \theta}{n})^{Z + m} } ] = 1$
	\end{itemize}
\end{frame}

\begin{frame}
	\frametitle{Formule}
	
	\begin{block}{Probabilité d'infection }
          $\sum_{k=0}^{l}\binom{n-k}{l-k}\frac{P_{k}^{n}}{\Phi( \frac{n-l}{n})^{k + m} }=\binom{n}{l}$
	\end{block}

\end{frame}

\begin{frame}
	\frametitle{Démonstration}
	\begin{itemize}
	\item $ \frac{{p_{i}}^{N}}{\binom{N}{i}}= \frac{{p_{i}}^{k}}{\binom{k}{i}}\left ( \exp(-(N-k)A^{k}|Z^{k}=i)  \right )$
	\item $\sum_{i=0}^{k} \frac{ \mathbb{E}[e^{-(N-k) A^{k}}|Z^{k}=i ]}{ \Phi( \frac{\lambda (N-k)}{n})^{i + m} } ]p{_{i}}^{k} = 1$
	\item $\binom{k}{i}/\binom{N}{i} = \binom{N-i}{k-i}\binom{N}{k}$
	\item $\sum_{k=0}^{l}\binom{N-i}{k-i}\frac{P_{k}^{n}}{\Phi( \frac{\lambda (N-k)}{n})^{i + m} }=\binom{N}{i}$
	\end{itemize}
\end{frame}


