\section{La construction de Selke}

Une construction équivalente, proposée par Sellke en 1983, se base sur l’idée d’affecter à chaque individu une capacité de résistance à l’épidémie et de définir, à partir du nombre d’infectés, une pression de l’épidémie et un critère d’infection sur la résistance de l’individu. Pour y arriver, les chaînes de Markov sont utilisées pour modéliser l’évolution de l’épidémie au fil du temps, en prenant en compte les transitions entre les différents états de l’épidémie (susceptible, infecté, rétabli). Les processus de Poisson sont utilisés pour modéliser le nombre d’infections qui se produisent dans un intervalle de temps donné, en supposant que les infections se produisent de manière aléatoire et indépendante les unes des autres.

De plus cette construction permet aussi d’identifier un paramètre important quand il s’agit de modéliser une épidémie : sa taille finale. Ici, l’expression de la taille finale de l’épidémie dépend du nombre de reproductions de base $R_0$, qui est le nombre moyen de personnes qu’un individu infecté infecte au début de l’épidémie. Plus le nombre de reproductions de base est élevé, plus la taille finale de l’épidémie sera grande. L’expression de la taille finale de l’épidémie peut être calculée à partir de l’équation $1 - \tau = e - R_0\tau$, où $\tau$ est la proportion de la population qui sera infectée à la fin de l’épidémie.


\subsection{Chaine de Markov}

Une chaîne de Markov est un processus aléatoire à temps discret dont la principale caractéristique est l’absence de mémoire et l’existence de probabilités de transition entre les états de la chaîne. Cela signifie que seul l’état actuel du processus a une influence sur l’état qui va suivre.

En considérant S un ensemble fini (ou dénombrable) et $(X_n)_{n \in \mathbb{N}}$ une suite de variables aléatoires définie sur le même univers $\Omega$ munit de la mesure de probabilité $\mathbb{P}$ et à valeurs dans S. $(X_n)$ chaîne de Markov homogéne si :

\begin{itemize}
    \item \textbf{Propriété de Markov} : $\forall n \in \mathbb{N}$ et $(x_i)^n \in S^n (i \in [[0, n+1]])$ \\
    $$ \mathbb{P}(X_{n+1} = x_{n+1} | X_{0:n} = x_{0:n}) = \mathbb{P}(X_{n+1} = x_{n+1} | X_n = x_n) $$
    \item \textbf{Homogénéité} : $\mathbb{P}(X_{n+1} = y | X_n = x)$ ne dépend pas de n, $\forall (x, y) \in S^2$. On note alors $p(x, y)$ cette probabilité.
\end{itemize}

La loi de $X_0$ est appelée loi initiale de la chaîne de Markov, $P = (p(x, y))_{x,y \in S}$ est appelée matrice de transition.

\subsection{Processus de Poisson}


